\documentclass[11pt,a4paper,french]{article}
\usepackage[utf8]{inputenc}
\usepackage{graphicx}
\usepackage{color}
\usepackage{listings}

\usepackage{color}

\usepackage[francais]{babel}

\usepackage{amsmath}
\usepackage{graphicx}
\usepackage{amssymb}
\usepackage{float}

\usepackage{hyperref}

\usepackage[sorting=none]{biblatex}
\addbibresource{ref.bib}

\definecolor{dkgreen}{rgb}{0,0.6,0}
\definecolor{gray}{rgb}{0.5,0.5,0.5}
\definecolor{mauve}{rgb}{0.58,0,0.82}

\usepackage{geometry}
\geometry{hmargin=2.5cm,vmargin=2.5cm}

\title{\LARGE{Rain Nowcasting} \\ [0.5cm]\Large{Projet PRAT} \\ [0.25cm]\large{Rapport bibliographique}}
\author{Lucas Beretti \\ Encadrant : Dominique Béréziat}
\date{25 octobre 2021}

\begin{document}

\maketitle

\tableofcontents

\newpage

\section{Introduction}

Tout au long de son histoire, l'homme a porté un intérêt certain au domaine de la prévision météorologique du fait de la forte dépendance de ses activités à celui-ci. Aujourd'hui encore, elle joue un rôle central dans de nombreux secteurs météo-sensibles comme l'agriculture, le transport, l'énergie, le tourisme ... Bien que les méthodes d'estimation de la météo se sont largement améliorées au cours des dernières décennies, la fiabilité de ces modèles est un sujet sur lequel les météorologues travaillent encore.

La prévision du taux de précipitation est une donnée cruciale pour l'ensemble des secteurs météo-sensibles. Ces prévisions sont effectuées par des modèles physiques intégrant divers phénomènes liés à la dynamique de l'atmosphère. Bien que ces méthodes apportent de bonnes prévisions de précipitation sur plusieurs jours, elles manquent encore de précision à très court terme, jusqu'à deux heures à l'avance. Cette problématique est plus communément appelée "Rain Nowcasting".

L'émergence des technologies d'apprentissage automatique, et plus précisément d'apprentissage profond, a amené certains chercheurs à s'intéresser au problème de prévision du taux de précipitation à court terme. 
Cet état de l'art a pour but de présenter un panel de méthodes qui ont été élaborées au cours des dernières années.



\section{Méthodes traditionnelles}

Un premier article parle de différentes méthodes traditionnelles (TREC, MOVA, ROVER) \cite{atmos8030048}. \newline

\noindent
Une librairie Python implémentant des méthodes de prédiction de pluie à court terme. \cite{article}

\section{Méthodes par apprentissage profond}

\subsection{Approches récurrentes}

\subsubsection{ConvLSTM}

L'approche récurrente par ConvLSTM représente la toute première méthode par Deep Learning sur le sujet. Cet article présente des améliorations vis-à-vis de l'approche ROVER basée sur le flot optique. \cite{shi2015convolutional}

\subsubsection{TrajGRU}

D'autres travaux ont été réalisés avec des approches récurrentes dans le but d'améliorer la solution ConvLSTM. Le majeur reproche qui lui était fait est l'invariance par translation introduite par l'opération de convolution qui ne permet pas de bien décrire les mouvements de masses nuageuses comme le fait une approche par flot optique. Par conséquent, le modèle TrajGRU apporte une solution à ce problème en déterminant de manière dynamique, par apprentissage, les connexions entre les couches cachées. \cite{shi2017deep}

\subsection{Approches U-Net}

Une approche U-Net à laquelle on veut comparer les résultats des autres approches. \cite{bouget:hal-03112093} \newline

Un second article compare une approche U-Net, similaire du papier précédent en architecture, au modèle TrajGRU. Les résultats obtenus par le U-Net offrent une meilleure précision à faible taux de précipitation mais souffrent à estimer fiablement un taux de précipitation plus élevé.  \cite{9508500}

\subsection{Approche par réseau de neurones génératif}

Comparaison d'un réseau GAN à plusieurs approches dont un U-Net. De meilleurs résultats sont obtenus pour des taux de précipitation plus importants et ce réseau préserve une meilleure enveloppe spatiale. \cite{ravuri2021skillful}

\section{Réalisation du projet}

\subsection{Données utilisées}

\subsection{Objectifs du projet}

Les objectifs de ce problème seront de :
\begin{itemize}
    \item Comparer les approches récurrentes et GAN au U-Net en \cite{bouget:hal-03112093}.
    \item Étudier l'effet de l'ajout de données de vent sur l'amélioration des prédictions. 
    \item Si l'approche génératrice apporte les résultats les plus satisfaisants, on pourra étudier d'autres approches avec un réseau générateur inspiré de l'architecture d'autres articles comme le U-Net par exemple. 
\end{itemize}

\subsection{Calendrier prévisionnel}

\section{Conclusion}

\newpage

\printbibliography[title=Références]

\end{document}
